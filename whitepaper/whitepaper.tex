\documentclass[table, twocolumn]{article}
\usepackage{amsmath}
\usepackage{hyperref}
\usepackage{geometry}
\usepackage[acronym]{glossaries}
\usepackage{pgfplots}
\usepackage{xcolor}
\pgfplotsset{compat=1.18}
\usetikzlibrary{arrows.meta}
\usetikzlibrary{intersections}

% Page options.
\pagecolor{white}
\color{black}
\geometry{left=35pt, top=50pt, bottom=50pt, right=35pt}

% Acronyms.
\newacronym{nft}{NFT}{Non-Fungible Token}
\newacronym{ai}{AI}{Artificial Intelligence}
\newacronym{pvp}{PvP}{Player vs Player}
\newacronym{vrf}{VRF}{Verifiable Random Function}

% Links.
\hypersetup{colorlinks=true, allcolors={blue}}

\title{%
  \Huge \texttt{trust16} \\ \vspace{10pt}
  \small \emph{A Blockchain-Based Game of Strategy and Cooperation}
}
\author{Code&Ledger}
\date{}

\begin{document}

\maketitle

\section{Overview} \label{sec:overview}

\texttt{trust16} bootstraps trust dynamics in a gamified environment using a
multi-state mechanism popularized by similar blockchain projects. The first state,
known as the Short Game Mode, uses an abridged \gls{pvp} format with multiple rounds
as described in section \ref{sec:short-game-mode}.

Once a player's reputation reaches a predefined value, a State Transition occurs
whereby they unlock access to the Long Game Mode, featuring more complex strategic
elements as outlined in section \ref{sec:long-game-mode}.

\section{Economic variables}

Price $p$ is defined per table \ref{tab:t-r-definitions} and equation
\ref{eqn:price-defined}.

\begin{table}[!htb]
  \centering
  \begin{tabular}{|c|c|c|}
    \hline \rowcolor{gray!20}
    Term          & Notation & Asset        \\ \hline
    TRUST token   & $t$      & TRUST        \\ \hline
    Rewards Pool  & $r$      & TRUST        \\ \hline
  \end{tabular}
  \caption{TRUST token and Rewards Pool definitions}
  \label{tab:t-r-definitions}
\end{table}

\begin{equation} \label{eqn:price-defined}
  p = \frac{t}{r}
\end{equation}

The economic variables in table \ref{tab:game-model-variables} fully specify the set of
numerical values required for the implementation, as derived in sections
\ref{sec:short-game-mode} through \ref{sec:tokenomics}.

\begin{table}[!htb]
  \centering
  \begin{tabular}{|c|c|}
    \hline \rowcolor{gray!20}
    Term                                   & Notation \\ \hline
    Player deposit                         & $d_p$    \\ \hline
    Rewards pool contribution              & $c_r$    \\ \hline
    Cooperation reward                     & $r_c$    \\ \hline
    Competition reward                     & $r_t$    \\ \hline
  \end{tabular}
  \caption{Economic variables for game modes}
  \label{tab:game-model-variables}
\end{table}

\section{Short Game Mode} \label{sec:short-game-mode}

The Short Game Mode in Trust16 is designed to create a fast-paced, strategic experience
that encapsulates the core dynamics of trust and betrayal. Here's a detailed breakdown
of how it works:

\subsection{Setup}
\begin{itemize}
    \item Each player deposits $d_p$ TRUST
    \item The rewards pool contributes $c_r$ TRUST
    \item Total Game Pool starts at $2d_p + c_r$ TRUST
\end{itemize}

\subsection{Rounds}
The game consists of 5 rounds. In each round:
\begin{itemize}
    \item Players simultaneously choose to either Cooperate (Green) or Compete (Red)
    \item Choices are revealed, and TRUST is redistributed based on the decisions
\end{itemize}

\subsection{TRUST Distribution}
TRUST distribution follows the pattern described in table \ref{tab:trust-distribution}.

\begin{table}[!htb]
  \centering
  \begin{tabular}{|c|c|c|}
    \hline \rowcolor{gray!20}
    Scenario        & Player 1                 & Player 2                 \\ \hline
    Green-Green     & $r_c$ from deposit       & $r_c$ from deposit       \\
                    & $+r_c$ from rewards pool & $+r_c$ from rewards pool \\ \hline
    Red-Green       & Previous balance         & $r_c$ from rewards pool  \\
                    & $+$ Player 2's balance   &                          \\
                    & $+r_t$ from Player 2     &                          \\ \hline
    Red-Red         & 0 TRUST                  & 0 TRUST                  \\ \hline
  \end{tabular}
  \caption{TRUST distribution based on player choices}
  \label{tab:trust-distribution}
\end{table}

\section{Long Game Mode} \label{sec:long-game-mode}

The Long Game Mode offers a more in-depth strategic experience:

\begin{itemize}
    \item Duration: Variable, based on player-set chat time
    \item Includes a chat phase for negotiation and strategy discussion
    \item Single round with higher stakes
    \item Complex reward distribution based on negotiation outcomes
\end{itemize}

\section{Technical Architecture}

\subsection{Blockchain Integration}
Trust16 leverages the Aptos blockchain for its smart contract functionality, ensuring
transparent and immutable game outcomes.

\subsection{Smart Contract}
The core smart contract handles functions described in table \ref{tab:smart-contract-functions}.

\begin{table}[!htb]
  \centering
  \begin{tabular}{|c|p{5cm}|}
    \hline \rowcolor{gray!20}
    Function        & Description                                        \\ \hline
    Player matching & Pairs players based on reputation and preferences  \\ \hline
    Bet escrow      & Securely holds player deposits during games        \\ \hline
    Outcome verification & Uses \glspl{vrf} to ensure fair play          \\ \hline
    Reward distribution  & Allocates TRUST based on game outcomes        \\ \hline
  \end{tabular}
  \caption{Smart contract core functions}
  \label{tab:smart-contract-functions}
\end{table}

\section{Tokenomics} \label{sec:tokenomics}

Trust16 introduces a unique, community-focused tokenomics model that provides direct
value to players while simultaneously enriching the game's reward ecosystem.

\subsection{Token Valuation and Purchase}
\begin{itemize}
    \item 1 TRUST = 0.05€ (or 1€ = 20 TRUST)
    \item Standard Purchase Example: For 10€, a total of 200 TRUST is minted
\end{itemize}

\subsection{Community-Boosting Dual Allocation}
When a player purchases TRUST tokens, the minted amount is equally split between the
player and the game's Rewards Pool:
\begin{itemize}
    \item Player Allocation: The player receives half of the total minted TRUST tokens
    \item Community Rewards Pool: The other half of the minted TRUST tokens is added directly to the game's Rewards Pool
\end{itemize}

\subsection{Rewards Pool Utilization}
The Rewards Pool, boosted by this dual allocation system, is used as described in
table \ref{tab:rewards-pool-utilization}.

\begin{table}[!htb]
  \centering
  \begin{tabular}{|c|p{5cm}|}
    \hline \rowcolor{gray!20}
    Utilization     & Description                                        \\ \hline
    Tournaments     & Fund larger prize pools for special events         \\ \hline
    Bonuses         & Provide rewards for consecutive cooperative plays  \\ \hline
    Creator Rewards & Offer incentives for content creation              \\ \hline
    Jackpot Events  & Create periodic high-value prize opportunities     \\ \hline
    Community Initiatives & Support engagement and growth activities     \\ \hline
  \end{tabular}
  \caption{Rewards Pool utilization}
  \label{tab:rewards-pool-utilization}
\end{table}

\section{Character Traits System}

\subsection{Core Traits}
The game maintains 16 standard character traits, each representing different strategic
approaches and playstyles.

\subsection{Seasonal Trait Variations}
To keep the game dynamic and encourage strategic adaptation:
\begin{itemize}
    \item Each season, 4-6 characters receive significant trait changes or "seasonal variants"
    \item Seasonal variants feature visual changes (themed skins) and adjusted trait behaviors
    \item Remaining characters receive minor trait tweaks to keep the meta fresh
    \item Occasional introduction of new characters (yearly) with potential retirement of underperforming ones
\end{itemize}

\section{Reputation System}

Trust16 implements a sophisticated Reputation System, represented by a soulbound token,
which reflects a player's standing within the community.

\subsection{Reputation Formula}
The reputation score (R) is calculated using the following formula:

\begin{equation}
R = \min(100, GP + CB + SS + CE + CP - RD)
\end{equation}

Where the components are defined as per table \ref{tab:reputation-components}.

\begin{table}[!htb]
  \centering
  \begin{tabular}{|c|p{5cm}|c|}
    \hline \rowcolor{gray!20}
    Component & Description                    & Max Points \\ \hline
    GP        & Games Played                   & 50         \\ \hline
    CB        & Cooperative Behavior           & 20         \\ \hline
    SS        & Successful Strategies          & 15         \\ \hline
    CE        & Community Engagement           & 10         \\ \hline
    CP        & Consistent Play                & 5          \\ \hline
    RD        & Reputation Decay               & Variable   \\ \hline
  \end{tabular}
  \caption{Reputation formula components}
  \label{tab:reputation-components}
\end{table}

\section{Conclusion}

Trust16 represents a pioneering effort in blockchain gaming, blending sophisticated
game theory with cutting-edge technology. By creating a platform that is simultaneously
a game, a social experiment, and a research tool, Trust16 aims to push the boundaries
of what's possible in the realm of decentralized applications.

As we move forward, we invite players, developers, researchers, and institutions to
join us in exploring the fascinating world of Trust16. Together, we can not only enjoy
a compelling game but also contribute to meaningful insights that could shape our
understanding of human behavior and social interactions in the digital age.

\end{document}
